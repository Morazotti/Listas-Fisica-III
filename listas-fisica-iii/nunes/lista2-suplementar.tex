% Created 2020-03-30 Mon 14:27
% Intended LaTeX compiler: pdflatex
\documentclass[11pt]{article}
\usepackage[utf8]{inputenc}
\usepackage[T1]{fontenc}
\usepackage{graphicx}
\usepackage{grffile}
\usepackage{longtable}
\usepackage{wrapfig}
\usepackage{rotating}
\usepackage[normalem]{ulem}
\usepackage{amsmath}
\usepackage{textcomp}
\usepackage{amssymb}
\usepackage{capt-of}
\usepackage{hyperref}
\usepackage{tikz}
\usepackage[portuguese]{babel}
\usepackage[margin=1in]{geometry}
\renewcommand{\thesection}{Questão}
\author{Nícolas André da Costa Morazotti}
\date{\today}
\title{Lista Suplementar 2}
\hypersetup{
 pdfauthor={Nícolas André da Costa Morazotti},
 pdftitle={Lista Suplementar 2},
 pdfkeywords={},
 pdfsubject={},
 pdfcreator={Emacs 26.3 (Org mode 9.1.9)}, 
 pdflang={Portuguese}}
\begin{document}

\maketitle
\section{1}
\label{sec:org81f442a}
Duas superfícies planas horizontais estão separadas por uma distância
\(d\). A superfície de cima tem densidade de carga \(\sigma\), e a de baixo tem
densidade \(-\sigma\). Desenhe as linhas de força nas três regiões do
espaço: entre as superfícies, acima da superior e abaixo da
inferior. Calcule a diferença de potencial entre as superfícies.

Vamos colocar o centro do sistema de coordenadas de forma que as placas
estejam em \(\pm d/2\). Como sabemos, seus campos são constantes. Então, a
placa de cima gera um campo \(\hat z\sigma/2\varepsilon_0\) para \(z>d/2\) e \(-\hat z\sigma/2\varepsilon_0\)
para \(z<d/2\). De forma similar, a placa de baixo gera um campo de mesmo
módulo, mas sentidos distintos: \(-\hat z\sigma/2\varepsilon_0\) para \(z>-d/2\) e \(\hat
z\sigma/2\varepsilon_0\) para \(z<-d/2\). Então, temos três regiões: abaixo da placa de
baixo, o campo é \(\sigma/2\varepsilon_0(-\hat z + \hat z)= 0\). Acima da placa de cima,
\(\sigma/2\varepsilon_0(\hat z - \hat z) - 0\). Entre as placas, \(\sigma/2\varepsilon_0 (-\hat z -\hat
z) = -\hat z\sigma/\varepsilon_0\). 

\begin{center}
\includegraphics[width=.9\linewidth]{./.ob-jupyter/e383ddf32f295ed09e95802221859516f2185389.png}
\end{center}

Para calcular a diferença de potencial entre as cargas, podemos utilizar
a integral \(V(A)-V(B)=-\int_B^A \mathbf E\cdot d\mathbf r\), onde \(A\) é um ponto
na placa de cima, \(B\) é um ponto na placa de baixo, com mesmo \((x,y)\)
para que possamos fazer um caminho retilíneo com \(d\mathbf r = -\hat
zdz\). A integração ocorre então de \(-d/2\) a \(d/2\). Então,
\begin{align*}
  V(cima) - V(baixo)
  &= -\int_{baixo}^{cima} \mathbf E\cdot d\mathbf{r}\\
  &= -\frac{\sigma}{\varepsilon_0}\int_{-d/2}^{d/2}dz\\
  &= -\frac{\sigma d}{\varepsilon_0}.
\end{align*}

\section{2}
\label{sec:orgdf3add6}
Um dipolo tem momento \(\mathbf p = (2qd)\hat z\). Desenhe as linhas de
força que saem e entram no dipolo. 

\begin{verbatim}
import numpy as np
import matplotlib.pyplot as plt
from matplotlib.pyplot import figure
figure(dpi=300,figsize=(6,3))

d=0.5
Z, Y = np.mgrid[-1.5:1.5:100j, -4:4:100j] 
U = Y/(Y**2 + (Z-d)**2)**(3/2) - Y/(Y**2 + (Z+d)**2)**(3/2) 
V = (Z-d)/(Y**2 + (Z-d)**2)**(3/2) - (Z+d)/(Y**2 + (Z+d)**2)**(3/2) 

plt.arrow(0,-d,0,2*d,length_includes_head=True, head_width=0.05,ec='k',fc='k')
plt.streamplot(Y, Z, U, V, density=[1.5, 1],color=V,cmap='autumn')
plt.title(r'Diagrama das linhas de força de um dipolo, de distância $2d$, em $x=0$')
plt.xlabel(r'y/2d')
plt.ylabel(r'z/2d')
plt.show()
\end{verbatim}

\begin{center}
\includegraphics[width=.9\linewidth]{./.ob-jupyter/6b94e8f43d9a94c4ddf2c52a2d9e8c248644fcdf.png}
\end{center}

\section{3}
\label{sec:org551563f}
Calcule o campo elétrico produzido pelo dipolo do item (2) no ponto com
coordenadas \((x,y,0)\). \emph{Sugestão: o campo elétrico no plano \(z=0\) é}
\emph{paralelo a \(\hat z\). Basta calcular \(\partial V / \partial z\).}

O potencial de um dipolo \(\mathbf p=2qd\hat z\) pode ser escrito como
\begin{align}
  V(\mathbf r) &= \frac{\mathbf p\cdot\mathbf r}{4\pi\varepsilon_0|\mathbf r|^3}\\
               &= \frac{pz}{4\pi\varepsilon_0|\mathbf r|^3}\\
               &= \frac{pz}{4\pi\varepsilon_0(x^2+y^2+z^2)^{3/2}}\\
  \frac{\partial}{\partial z}V(\mathbf r) &= \frac{p}{4\pi\varepsilon_0(x^2+y^2+z^2)^{3/2}} +
                              \frac{pz}{4\pi\varepsilon_0}\frac{\partial}{\partial
                              z}(x^2+y^2+z^2)^{-3/2}\\  
               &= \frac{p}{4\pi\varepsilon_0(x^2+y^2+z^2)^{3/2}} +
                              \frac{pz}{4\pi\varepsilon_0}\frac{(-6)z}{2(x^2+y^2+z^2)^{5/2}}\\
               &= \frac{p}{4\pi\varepsilon_0(x^2+y^2+z^2)^{3/2}} -
                              \frac{3pz^2}{4\pi\varepsilon_0(x^2+y^2+z^2)^{5/2}}.\label{eq:3}
\end{align}
No ponto \(\mathbf r=x\hat x+y\hat y\), o segundo termo se cancela ao
fazer \(z=0\). O campo elétrico é então
\begin{align}
  \mathbf E(x,y,0) = -\hat z \frac{qd}{2\pi\varepsilon_0(x^2+y^2)^{3/2}}.
\end{align}
\section{4}
\label{sec:orga9aa4f9}
Calcule o campo elétrico produzido pelo dipolo do item (2) no ponto com
coordenadas \((0,0,z)\) a partir da lei de Coulomb aplicada a cada uma das
cargas do dipolo.

Colocando a carga \(-2q\) na origem e a carga \(2q\) a uma distância \(d\) da
carga negativa, o campo elétrico, pela lei de Coulomb, é
\begin{align}
  \mathbf E(\mathbf r) &= \frac{2q}{4\pi\varepsilon_0}\left\{-\frac{\mathbf r}
                         {(x^2+y^2+z^2)^{3/2}}
                         +\frac{\mathbf r - d\hat z}
                         {[x^2+y^2+(z-d)^2]^{3/2}}\right\}\\
  \mathbf E(0,0,z) &= \frac{2q}{4\pi\varepsilon_0}\left[
                     -\frac{z\hat z}{z^3}+\frac{(z-d)\hat z}{(z-d)^3}
                     \right]\\
                       &= \frac{q\hat z}{2\pi\varepsilon_0}\left[
                         -\frac{1}{z^2}+\frac{1}{(z-d)^2}
                         \right].\label{eq:2}
\end{align}
Podemos expandir o termo \((z-d)^{-2}\), utilizando \(d/z\ll 1\):
\begin{align}
  (z-d)^{-2} = \frac {1}{z^2}(1-d/z)^{-2} \approx \frac{1}{z^2}(1+2 d/z) = \frac 1{z^2}+2\frac d{z^3}.
\end{align}
Substituindo na equação \ref{eq:2}, temos
\begin{align}
  \mathbf E(0,0,z) &\approx \frac{q\hat z}{2\pi\varepsilon_0}\left[
                         -\frac{1}{z^2}+\frac{1}{z^2}
                         +2\frac d{z^3}
                         \right]\\
                   &= \hat z\frac{qd}{\pi\varepsilon_0z^3}.\label{eq:5}
\end{align}

\section{5}
\label{sec:org763e9c4}
Calcule o campo elétrico produzido do item (2) no ponto com coordenadas
\((0,0,z)\) a partir da expressão para o potencial do dipolo.

A partir do potencial do dipolo, temos a expressão \ref{eq:3}. Veja que as
componentes \(\hat{x}\) e \(\hat{y}\) do gradiente são:
\begin{align}
  \frac{\partial}{\partial x}V(x,y,z) &= \frac{\mathbf p}{4\pi\varepsilon_0}\cdot\frac{\partial}{\partial
                          x}\frac{x\hat x+y\hat y+z\hat z} {(x^2+y^2+z^2)^{3/2}}  \\
                        &=\frac{\mathbf p\cdot \mathbf r}{4\pi\varepsilon_0}
                          \left[-\frac{6x}{2(x^2+y^2+z^2)^{5/2}}\right]
                          + \frac{p_x}{4\pi\varepsilon_0(x^2+y^2+z^2)^{3/2}}.
\end{align}
Como o dipolo aponta na direção \(z\), \(p_x \equiv 0\). O primeiro termo,
com \(x=0\), é \(0\). De maneira similar, poderíamos ter calculado a
componente \(\hat y\) do gradiente. A diferença é que teríamos \(6y\) ao
invés de \(6x\), que também se anula, e \(p_y\), que também é \(0\). Então só
temos a componente \(z\) do campo, que é a componente obtida na equação \ref{eq:3}.
Calculando tal equação no ponto \((0,0,z)\)
\begin{align}
  \frac{\partial}{\partial z}V(0,0,z) &= \frac{2qd}{4\pi\varepsilon_0 z^3}-\frac{3\cdot2qdz^2}{4\pi\varepsilon_0z^5}\\
                        &= \frac{qd}{2\pi\varepsilon_0 z^3}-\frac{3qd}{2\pi\varepsilon_0z^3}\\
                        &= \frac{qd}{2\pi\varepsilon_0 z^3}(1-3)\\
                        &= -\frac{qd}{\pi\varepsilon_0 z^3}.\label{eq:4}
\end{align}
Como \(\mathbf E=-\nabla V\), trocamos o sinal da equação \ref{eq:4} e chegamos à
equação \ref{eq:5}.
\section{6}
\label{sec:orgae1f32e}
Uma esfera de raio \(R\) tem densidade volumétrica de carga uniforme,
igual a \(\rho\). Encontre o potencial eletrostático em função da distância
\(r>R\), medida a partir do centro da esfera. \emph{Sugestão: Fora da esfera, o}
\emph{potencial equivale ao da carga da esfera concentrada em seu centro.}

Para fora da esfera, podemos utilizar como se o potencial fosse de uma
carga pontual centrada na origem. A carga total da esfera é \(Q = 4\pi R^3
\rho/3\). Assim,
\begin{align}
  V(r) &= \frac{Q}{4\pi\varepsilon_0}\frac{1}{r}\\
       &= \frac{\rho}{3\varepsilon_0}\frac{R^3}{r}.
\end{align}

\section{7}
\label{sec:org68be1b2}
Repita o problema anterior, mas agora calcule o potencial em função da
distância \(r<R\). \emph{Sugestão: Dentro da esfera, o campo elétrico é}
\emph{\(E(r)=(1/3\varepsilon_0)\rho r\). A diferença de potencial entre o ponto a distância}
\emph{\(r\) do centro e a superfície da esfera é \(\int_r^R E(r')dr'\).}

Para calcular a diferença de potencial entre um ponto a uma distância
\(r<R\) do centro da esfera, vamos substituir o campo elétrico
\(E(r)\). Vamos utilizar que o potencial elétrico é um campo escalar
\textbf{contínuo} no espaço. Assim, podemos afirmar que \(V(R) = V_{fora}(R)\).
\begin{align}
  V(r) - V(R) &= \int_r^R E(r')dr'\\
  V(r) - \frac{\rho R^2}{3\varepsilon_0}
              &= \frac {\rho}{3\varepsilon_0}\int_r^R r'dr'\\
              &= \frac {\rho}{3\varepsilon_0}\frac{r'^2}2\Big\vert_r^R\\
              &= \frac {\rho}{6\varepsilon_0}(R^2-r^2)\\
  V(r) &= \frac{\rho}{3\varepsilon_0}R^2 + \frac {\rho}{6\varepsilon_0}R^2-\frac{\rho}{6\varepsilon_0}r^2\\
              &= \frac{\rho}{2\varepsilon_0}R^2 -\frac{\rho}{6\varepsilon_0}r^2.
\end{align}

\begin{center}
\includegraphics[width=.9\linewidth]{./.ob-jupyter/ac6eb2316473ad7ba6d522e3782925c97b761550.png}
\end{center}

\section{8}
\label{sec:orgcc613a3}
Uma barra cilíndrica metálica infinita tem raio \(a\) e densidade
superficial de carga \(\sigma\). Encontre o campo elétrico num ponto P fora da
barra, a uma distância \(r>a\) do eixo da barra.

\begin{figure}[h!]
  \centering
  \begin{tikzpicture}
    \draw (0.5,2)  arc (0:360:0.5 and 0.2);
    \draw[dashed] (0.5,-2) arc (0:180:0.5 and 0.2);
    \draw (0.5,-2) arc (0:-180:0.5 and 0.2);
    \draw (0.5,2) -- (0.5,-2);
    \draw (-0.5,2) -- (-0.5,-2);
    \draw[dashed] (0,2) -- (0,-2.2);
    \draw (0,2) -- (0,2.5);
    \draw (0,-2.2) -- (0,-2.5);
    \filldraw[black] (1.5,0) circle (1pt) node[right] {$P$};
    \draw[<->] (0,2.3) -- (0.5,2.3) node[above left] {$a$};
    \node[left] at (-0.5,0) {$\sigma$};
  \end{tikzpicture}
  \caption{Diagrama das questões 8 e 9.}
  \label{fig:ex-8}
\end{figure}

Para encontrar o campo elétrico no ponto \(P\), podemos utilizar uma
superfície gaussiana cilíndrica de altura \(L\) e raio \(r\). Pela lei de
Gauss,
\begin{align}
  E 2\pi rL &= \frac{\sigma 2\pi a L}{\varepsilon_0}\\
  E &= \frac{\sigma a}{\varepsilon_0 r}\\
  \mathbf E &= \frac{\sigma a}{\varepsilon_0 r}\hat r\label{eq:1}.
\end{align}

\section{9}
\label{sec:orgece47e6}
A partir do resultado da questão anterior, calcule o potencial no mesmo
ponto P, isto é, a uma distância \(r<a\) do centro da barra. \emph{Sugestão:}
\emph{Tome como referência um ponto \(\bar O\) na superfície da barra.}

Utilizando o campo da equação \ref{eq:1}, podemos utilizar um caminho
radial que sai da casca cilíndrica e vai até \(P\), de forma que \(d\mathbf
r = \hat r dr\). Colocando o potencial na casca como nulo (afinal, não
podemos zerar o potencial com \(r\to\infty\)), 
\begin{align}
  V(r) - \underbrace{V(a)}_{\equiv 0}
  &= -\int_a^r \mathbf E(r')\cdot d\mathbf r'\\
  V(r) &= -\frac{\sigma a}{\varepsilon_0}\int_a^r \frac{dr'}{r'}\\
  &= -\frac{\sigma a}{\varepsilon_0}\ln(r')\Big\vert_a^r\\
  &= -\frac{\sigma a}{\varepsilon_0}\ln\left(\frac{r}{a}\right).
\end{align}
Internamente ao fio condutor, o campo elétrico é nulo, o que implica que
o potencial elétrico é constante.

\begin{center}
\includegraphics[width=.9\linewidth]{./.ob-jupyter/ec2a52051c407f5b9a9a73e4f66607adbbddb7b1.png}
\end{center}

\section{10}
\label{sec:org6f1f875}
Um dipolo está imerso num campo elétrico uniforme \(\mathbf E=E_0 \hat
z\). O centro do dipolo está na origem do sistema de coordenadas. O
momento do dipolo está no plano \(yz\) e forma um ângulo \(\theta\) com a direção
\(\hat z\). Calcule o torque que o campo elétrico produz sobre o
dipolo. \emph{Sugestão: Calcule a força que cada carga do dipolo sofre, devida}
\emph{ao campo elétrico, e calcule o torque, que é a soma de} \(\mathbf r_j
\times \mathbf F_j\), \emph{onde \(\mathbf r_j\) é a posição de cada carga e} \(\mathbf
F_j\) \emph{é a força sobre ela.}

\begin{figure}[h!]
  \centering
  \begin{tikzpicture}
    \draw[thick,->] (0,0) -- (0,2) node[above] {$z$};
    \draw[thick,->] (0,0) -- (2,0) node[above] {$y$};
    \node[above] at (-1.2,1) {$\mathbf E$};
    \foreach \x in {-2.2,...,2.2}
        \draw[->] (\x,-0.5) -- (\x,1); 
    \draw[thick,->] (0,0) -- (0.75,1.5) node[right] {$\mathbf p$};
    \draw (0,0.5) arc (90:60:0.5);
    \node[above right] at (-0.1,0.5) {$\theta$};
    \filldraw[black] (0,0) circle (1.5pt);
  \end{tikzpicture}
  \caption{Diagrama do exercício 10.}
  \label{fig:ex-10}
\end{figure}
Vamos chamar a distância entre as cargas de \(d\), e as cargas de \(q\). Uma
vez que o centro do dipolo se encontra na origem do sistema de
coordenadas, a posição da carga positiva é
\((0,d\sin\theta/2,d\cos\theta/2)\) e a da carga negativa é
\((0,-d\sin\theta,-d\cos\theta)\). A força que o campo elétrico \(\mathbf
E\) faz sobre a carga positiva é \(\mahtbf F_+ = \hat z E_0 q\) e sobre a
carga negativa é \(\mahtbf F_- = -\hat z E_0 q\). O torque que o campo faz
sobre as cargas pode ser escrito como
\begin{align}
  \boldsymbol \tau &= \mathbf{r}_+\times\mathbf{F}_++\mathbf{r}_-\times\mathbf{F}_-\\
                &= \frac{E_0qd}{2}[(\sin\theta\hat y+\cos\theta\hat z)\times\hat z +
                  (-\sin\theta\hat y-\cos\theta\hat z)\times(-\hat z)]\\
                &= \frac{E_0qd}{2}[\sin\theta\hat x +
                  \sin\theta\hat x]\\
                &= \hat x E_0 qd \sin\theta\\
                &= \hat x E_0 p \sin\theta\\
                &= \mathbf p\times\mathbf E.
\end{align}
\end{document}