% Created 2020-04-03 Fri 16:30
% Intended LaTeX compiler: pdflatex
\documentclass[11pt]{article}
\usepackage[utf8]{inputenc}
\usepackage[T1]{fontenc}
\usepackage{graphicx}
\usepackage{grffile}
\usepackage{longtable}
\usepackage{wrapfig}
\usepackage{rotating}
\usepackage[normalem]{ulem}
\usepackage{amsmath}
\usepackage{textcomp}
\usepackage{amssymb}
\usepackage{capt-of}
\usepackage{hyperref}
\usepackage{tikz}
\usepackage[portuguese]{babel}
\usepackage[margin=1in]{geometry}
\date{\today}
\title{Rotacionais e gradientes}
\hypersetup{
 pdfauthor={},
 pdftitle={Rotacionais e gradientes},
 pdfkeywords={},
 pdfsubject={},
 pdfcreator={Emacs 26.3 (Org mode 9.1.9)}, 
 pdflang={Portuguese}}
\begin{document}

\maketitle
Podemos entender o que ocorre ao calcularmos o rotacional de um campo da
seguinte maneira: suponha um campo \(\mathbf A = xy(\hat x+\hat y+\hat
z)\). Primeiramente, definimos o campo:

\begin{verbatim}
import numpy as np
from sympy import *
from sympy.physics.vector import ReferenceFrame, curl, divergence, gradient

S = ReferenceFrame('S')
A = (S[0]*S[1])*(S.x+S.y+S.z)
vars = [S[0],S[1],S[2]]
A
\end{verbatim}

$\displaystyle S_{x} S_{y}\mathbf{\hat{s}_x} + S_{x} S_{y}\mathbf{\hat{s}_y} + S_{x} S_{y}\mathbf{\hat{s}_z}$

Veja que podemos calcular o rotacional de \(\mathbf A\) referente ao
referencial \(S\) com \texttt{curl(A,S)}:
\begin{verbatim}
curl(A,S)
\end{verbatim}

Desejamos plotar o campo \(\mathbf A\) e seu rotacional num plano \(z=0\). 
\begin{verbatim}
import matplotlib.pyplot as plt
from matplotlib.pyplot import figure
figure(dpi=300)

X,Y = np.mgrid[-4:4:100j,-4:4:100j]
U = lambdify(vars,A.dot(S.x),modules='numpy')
V = lambdify(vars,A.dot(S.y),modules='numpy')
plt.streamplot(Y,X, U(X,Y,0), V(X,Y,0))
plt.title(r'Campo $\mathbf{A} = xy(\hat{x}+\hat{y}+\hat{z})$')
plt.xlabel(r'x')
plt.ylabel(r'y')
plt.show()
\end{verbatim}

\begin{center}
\includegraphics[width=.9\linewidth]{./.ob-jupyter/43d316de40283c44695830cfba0f8ecfad906438.png}
\end{center}
\begin{verbatim}
U = lambdify(vars,curl(A,S).dot(S.x),modules='numpy')
V = lambdify(vars,curl(A,S).dot(S.y),modules='numpy')
figure(dpi=300)
plt.streamplot(Y,X, U(X,Y,0), V(X,Y,0))
plt.title(r'$\nabla\times\mathbf{A}$')
plt.xlabel(r'x')
plt.ylabel(r'y')
plt.show()
\end{verbatim}

Se tivéssemos definido o campo \(\mathbf A\) como um gradiente de um campo
escalar \(f=xy\),
\begin{verbatim}
f = S[0]*S[1]
A = gradient(f,S)
A
\end{verbatim}

$\displaystyle S_{y}\mathbf{\hat{s}_x} + S_{x}\mathbf{\hat{s}_y}$

\begin{verbatim}
U = lambdify(vars,A.dot(S.x),modules='numpy')
V = lambdify(vars,A.dot(S.y),modules='numpy')
figure(dpi=300)
plt.streamplot(Y,X,U(X,Y,0), V(X,Y,0))
plt.title(r'$\mathbf{A}=\nabla f$')
plt.xlabel(r'x')
plt.ylabel(r'y')
plt.show()
\end{verbatim}

\begin{center}
\includegraphics[width=.9\linewidth]{./.ob-jupyter/579c4a13235dd668ec5ce8d55cf2e8077e16ab42.png}
\end{center}
De tal modo que \(\nabla\times A=0\):
\begin{verbatim}
curl(A,S)
\end{verbatim}

$\displaystyle 0$

\section{Integration}
\label{sec:org9652966}
\begin{verbatim}
import matplotlib.pyplot as plt
import numpy as np
from sympy import *
from sympy.physics.vector import ReferenceFrame
from sympy.vector import Del

x,y,z = symbols('x y z')
nabla = Del()
S = ReferenceFrame('S')
r_vec = sum([S[indice]*versor for indice,versor in enumerate(S)])
r = r_vec.magnitude()
E = r_vec/r**3
dx = (S.x+S.y+S.z)

Integral(Integral(Integral(-E.dot(dx),(S[0],oo,x)),(S[1],oo,y)),(S[2],oo,z)).doit()

\end{verbatim}

$\displaystyle \int\limits_{\infty}^{z} \left(- \int\limits_{\infty}^{y} \frac{S_{y} x}{\sqrt{x^{2} + \operatorname{polar\_lift}{\left(S_{y}^{2} + S_{z}^{2} \right)}} \operatorname{polar\_lift}{\left(S_{y}^{2} + S_{z}^{2} \right)}}\, dS_{y} - \int\limits_{\infty}^{y} \frac{S_{z} x}{\sqrt{x^{2} + \operatorname{polar\_lift}{\left(S_{y}^{2} + S_{z}^{2} \right)}} \operatorname{polar\_lift}{\left(S_{y}^{2} + S_{z}^{2} \right)}}\, dS_{y} - \int\limits_{\infty}^{y} - \frac{1}{\sqrt{S_{y}^{2} + S_{z}^{2} + x^{2}}}\, dS_{y}\right)\, dS_{z}$
\end{document}